\documentclass[a4paper,11pt]{article}

\usepackage{revy}
\usepackage[utf8]{inputenc}
\usepackage[T1]{fontenc}
\usepackage[danish]{babel}

\revyname{Biorevy}
\revyyear{2014}
\version{1}
\eta{$2$ minutter}
\status{Færdig}
\responsible{Jean}

\title{Der er et institut}
\author{Marcus G}
\melody{Adam Oehlenschläger: ``Originaltitel''}

\begin{document}
\maketitle

\begin{roles}
    \role{S1}[Frederikke] Sanger 1
    \role{S2}[Anna Aa] Sanger 2
    \role{S3}[Søren Gade] Sanger 3
\role{S4}[Sus] Sanger 4
\role{S5}[Helge To-Polusen] Sanger 5
    \role{K1}[Nathalia] Kor 1
    \role{K2}[Flora] Kor 2
    \role{K3}[Anne-Mette] Kor 3
\role{K4}[Charlotte] Kor 4
\role{K5}[Jean] Kor 5
\role{AV}[AV] Video i start og slut
\end{roles}

\begin{props}
    \prop{Trappe}[Person, der skaffer] Trappe fra sikkerhedsintroen
\end{props}

\scene

Noter til bandet

Nummeret har samme form som originalen:

Scene \& stemning:
Der starter en sanger på scenen. Så kommer der to sangere ind sammen. Og så vælter det ind med kor (alle revyister!)
Evt. kan sangen også laves til kun at være kor (mere skønsang og harmonier i revyen!).

Bemærkninger til Teknikken:
A/V: Helge synger første vers på A/V. Når sangen slutter, kommer Helge
på A/V igen.


\begin{song}
\sings{S5} 
Der er et institut,
det står i Nørre Campus
nær Rigets hospital
nær Rigets hospital
Et stenkast væk fra HCØ
Her lever biologer
Hvert mandfolk og hver mø
Hvert mandfolk og hver mø

\sings{?}
Der sad i fordums tid
En mand i grønne waders
Og en i kittel hvid
Og en i kittel hvid
De ville vide alt om liv
Fra mus til mastodonter
Fra andemad til siv
Fra andemad til siv

Og folket det er skønt
Der's viden er nødvendig
Og studiet det er grønt!
Og studiet det er grønt!
Vi redder verden nat og dag
Vi kæmper for miljøet
Vi elsker vores fag
Vi elsker vores fag

Hil hver en biolog!
Hil dette store studie;
et folk i Darwins tro!
et folk i Darwins tro!
Vort stolte studie skal bestå,
så længe bøgen spejler
sin top i bølgen blå
sin top i bølgen blå

\end{song}

\end{document}

