\documentclass[a4paper,11pt]{article}

\usepackage{revy}
\usepackage[utf8]{inputenc}
\usepackage[T1]{fontenc}
\usepackage[danish]{babel}

\revyname{Biorevy}
\revyyear{2014}
% HUSK AT OPDATERE VERSIONSNUMMER
\version{1}
\eta{$lol$ minutter}
\status{Ikke færdig}
\responsible{Jennie}

\title{Sofie Carsten}
\author{Jakob, Danny}

\begin{document}
\maketitle

\begin{roles}
    \role{S/C}[Jennie] Sofie og Carsten (én person)
    \role{SP}[Mia] Speaker
\end{roles}

\begin{props}
    \prop{Bord}[DIKU] Rekvisitforklaring
    \prop{Stol}[DIKU] Rekvisitforklaring
    \prop{PC}[Jennie] Rekvisitforklaring
\end{props}

\scene 
Bemærkninger til Teknikken:
lyserødt socialist lys til Sofie, og ONDT mørkeblåt liberalist lys hos Carsten (Radikal! lol)


\begin{sketch}

\scene
Sofie kommer gående ind på scenen fra den ene side, med siden til (så man ikke kan se Carsten).
Hun vil gerne fortælle om alle de gode ting hun gerne vil gøre for studerende. Og hun forstår ikke helt al vreden og frustrationen kommer fra. Hun vil jo bare gøre det godt for alle. Medens hun fortæller om det, afbrydes hun pludselig af Carsten ( hun vender sig om så hun har den anden side til, nu er hun Carsten) Carsten vil kun uddanne eliten! ”Lad de kloge være kloge og de dumme være dumme”. ”Spild ikke på penge på dumme mennesker der bare sidder på job-centre eller åbner gallerier.” 

\says{SP}[Voice-over] Velkommen til Undervisningsminister Sofie Carsten Nielsen

\says{S}[Glad og optimistisk] Ej hvor jeg glæder mig til de næste finanslovs forhandlinger. Mon ikke der falder en stor sum penge af til alle de dejlige, hårdtarbejdende studerende. Det burde kunne rette op på alle de sure miner der har spredt sig. Jeg forstår i virkeligheden ikke rigtig hvor de kommer fra. Jeg --Uddannelsesminister Sofie Nielsen-- vil jo bare sørge for at alle kan få den uddannelse de gerne vil have. ” ” Det er så vigtigt at fx humaniora....”

\scene Sofie-Carsten vender sig hurtigt om og afslører Carsten -- lyset bliver blåt og ondt

\says{C}AFSKAFFES!! Aldrig er der blevet brugt så mange penge på ubrugelige studier! Og på de Naturvidenskabelig studier, er der alt for mange studerende. Der er jo ingen grund til at uddanne andre end de klogeste. Ellers spilder jeg, Uddannelsesminister Carsten Nielsen, jo penge på dumme mennesker der alligevel bare kommer til at sidde på job-centre eller åbne gallerier.”

\says{S} Der har jo også været så mange der gerne har villet kunne få overført deres relevante fag hvis de nu havde taget en forkert uddannelse. Men med alt det administration er der jo så svært for de stakkels studerende som i forvejen har travlt. Hmm måske man kunne\ldots

\says{C} Tvinge dem til at alle deres fag bliver overført lige meget hvad. Så kan de lære at vælge rigtigt i første forsøg og så vil de dumme studerende der ikke kan finde ud af hvad de vil komme ud af mine elite universiteter i en fart.

\says{S} Nå men vi skal i hvert fald have flere højt uddannede, det kunne vi måske løse ved at...”

\says{C} ”\ldots SKÆRE I ANTALLET AF STUDIEPLADSER, så mit universitet kun er for eliten! - åh hvor er det dog pisse irriterende, at jeg hele tiden bliver afbrudt af den dumme kælling! Det bare så typisk politik, at man ikke kan få et ord indført – \act{Carsten imiterer Sofie} ´´Uhh, jeg er Sofie Nielsen, og jeg synes, at der skal være plads til alle, bla bla bla.. Ingen er mere værd end andre, møhnøhnøh, Alle er lige, min bare røv!'' 

\says{S} ”Man skal jo huske, at der skal være plads til alle! Der er ikke nogen mennesker der er mere værd end andre. Alle er lige og har ret til en uddannelse\ldots

\says{C}[mere og mere frustreret]ØØÅÅåååHHH! Det går simpelthen ikke mere det her, jeg kan jo ikke få ført noget igennem hvis jeg hele tiden bliver afbrudt af den lalleglade hippie! Men som mor altid har sagt: ´´Hvis du møder en forhindring på din vej, så slå den ihjel!''

\scene{Der er et bord, og på bordet ligger en gun, eller på en eller anden måde får han i hvert fald fat i en.}

\says{C} Hmm, ja hvorfor ikke \ldots

\scene Carsten tager pistolen og skyder sig selv i hovedet
\scene BANG og blink han falder ned! 
\says{S} Neeeeeeej!
\scene Carsten rejser sig op og er nu bare Carsten.
\says{C} Mwhahahaha! Nu er jeg endelig fri, nu er jeg \textsc{carsten-carsten Nielsen}! Nu kan jeg omsider OVERTAGE VERDENSHERREDØMMET! Mwahahahaha!
lys ned?

\scene{Lys ned.}

\end{sketch}
\end{document}
