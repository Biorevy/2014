\documentclass[a4paper,11pt]{article}

\usepackage{revy}
\usepackage[utf8]{inputenc}
\usepackage[T1]{fontenc}
\usepackage[danish]{babel}

\revyname{Biorevy}
\revyyear{2014}
% HUSK AT OPDATERE VERSIONSNUMMER
\version{0.1}
\eta{$lol$ minutter}
\status{Ikke færdig lol}
\responsible{Jennie}

\title{Statistiksketchen (lol)}
\author{Markus Drag (lol)}

\begin{document}
\maketitle

\begin{roles}
    \role{P}[Helene] Parametrisk test, ``Kresten-type''
    \role{NP}[Sus] Nonparametrisk test, praktisk gris-type
    \role{S}[Mia] Kvindelig specialistuderende
\end{roles}

\begin{props}
    \prop{Excelcomputer}[Rekvisitgruppen] Flyttekassestørrelse

\end{props}




\begin{sketch}

\scene{En kvindelig specialestuderende sidder på sit specialekontor og skal behandle sit data.}
\scene{Lys op.}
\says{S} Ahhh, puuha! Så er det i gang med at bearbejde data til mit speciale. 

\says{S} Men – hvilken slags metode skal jeg mon vælge? Der findes jo to slags. Den parametriske metode --

\scene{Parametriske metode kommer ind på scenen}

\says{S} - som har masser af styrke -

\scene P flekser muskler.

\says{S} - og nøjagtigt -

\scene P skyder med bue og pil lige mod en målskive. Ligemeget om han rammer rigtigt eller ej, men det er altid med en målskive man eksemplificerer nøjagtighed og præcision.

\says{S} Ja, det er en rigtigt god metode! Den vælger jeg!

\says{P} \act{rømmer sig}

\says{S} Ja?

\says{P} Altså, hør lige engang, søde! Jeg er den parametriske metode! Og self vælger du mig, søde \act{Flexer mod publikum og blinker}

\says{S} Hvor fedt! Her er mine data.

\scene{Param metode kigger på kvindelig studerendes ``data''.}

\says{P} Hey hey hey, tror du selv jeg gider lave dit data? Det er jo SLET ikke normalfordelt, nærmere OVERALTfordelt \act{Param simulerer en slags fed dame/mand med sine hænder med helt forkert fordeling.}

\says{P} Og varianserne er jo fuldstændig hulter til bulter! \act{Simulerer med hænderne at håret sidder mega dårligt.}


\says{P}: Sorry, det bliver ikke os to! \act{stiller sig over i den anden side af scenen, tager diadem på og gør sig lækker.}

\says{S} Ok! Jamen så må jeg jo bruge en non-parametrisk metode.



\says{NP} \act{Non-param kommer ind på scenen} Skulle nogen bruge en hjælpende hånd?

\says{S} Den non-parametriske metode! Hmm... Den har jo ikke meget styrke

\scene{NP kigger på sine små muskler. Ikke imponerende.}

\says{S} og egentlig heller ikke ligefrem imponerende potens....

\scene{NP kigger ned og er flov}

\says{NP} Men tilgengæld stiller jeg ingen krav! Jeg tager bare alt data, ligemeget hvilken fordeling og varianser!! Wuaah [Ole Wedel lyden]

\act {Tager en hammer fra værktøjsbæltet og fikser et par tal sammen (rekvisitter)
Bukker sig ned og en kæmpe håndværkerrøv kommer til syne.}

\says{S} Jo altså ja. [lidt beskæmmet af røven]. Jaa jooo... Det kan vel godt gå an. 

\scene{NP fikser lidt forskelligt og er lidt ucharmerende håndværkertype.}

\says{S} Men jeg forstår bare ikke, mit data er det allerfineste klimaændrings-data -


\says{P+NP} Klimaændringer!? Det mest prestigiøse data man OVERHOVEDET kan få!

\says{P} Må jeg ikke nok teste? Lad mig bruge en ANOVA! Den tager du ikke fejl af!

\act{Tager en ANOVA test op som er en stor kastestjerne der står ANOVA på.}

\says{NP} Nej mig! PLEASE! Lad mig bruge en alsidig Kruskal-Wallis!

\act{Tager en KruskalWallis test op.}

\says{S} Ehm ok. Så. Jamen lad mig bruge jer begge to!!

\scene{S får lokkende (kvindelig charme) begge tests hen til sig.}

\scene{De to metoder kommer hen til stud og er glade og begejstret.}

\says{S} Så! Hop herned så jeg kan beregne jer! \act{Hiver Excel ark /computer op og lokker med fingeren.}

\says{P+NP} EXCEL??? NEEEEEEEJ!

\scene Rødt lys

\scene{Lys ned. lol}

\end{sketch}
\end{document}
