\documentclass[a4paper,11pt]{article}

\usepackage{revy}
\usepackage[utf8]{inputenc}
\usepackage[T1]{fontenc}
\usepackage[danish]{babel}

\revyname{Biorevy}
\revyyear{2014}
% HUSK AT OPDATERE VERSIONSNUMMER
\version{1.1}
\eta{$4$ minutter}
\status{Færdig -- skal strammes lidt op}
\responsible{Markus Back}

\title{Fagtermman}
\author{Danny, Thomas BT}

\begin{document}
\maketitle

\begin{roles}
    \role{FM}[Danny] Fagtermman
    \role{FD}[Valdbjørn] Fejedreng
    \role{GH}[Simon] Grønthandler
    \role{EW}[Markus Back] Eske Willerslev -- Fagtermmans sidekick
    \role{MS}[Jennie] Mystisk Stemme (Hemmeligt Sofie-Carsten) (Voice-over)
    \role{FT}[Hygge] Eske Willerslevs fortællerstemme (Voice-over)
    \role{SN}[Thomas BT] Sceneninja
\end{roles}

\begin{props}
    \prop{Boder}[Person, der skaffer] Frugt- \& grøntboder
    \prop{Telefon}[Person, der skaffer] Telefon
    \prop{Røg}[Person, der skaffer] Røg, m/u lys
    \prop{Superheltedragt}[Person, der skaffer] Fagtermmans Kostume
\end{props}


\begin{sketch}

\scene Grønthandlerbutikken består af nogle stande med frugt og grønt, hvor der står skilte som er i konflikt med fagtermerne ( Jordbær, kirsebær, rodfugter, grøntsager adskilt fra frugt osv. ) de kan sagtens være flade
\scene Fagtermand, iført sit kostume og med biologiske fagtermer i hånden, kommer snigende ind i butikken, hvor en fejedreng står med ryggen til og fejer. Da han ser skiltene opfører han sig vildt og voldsomt. Fejedrengen opdager ham, pga. postyret:

\says{FD} Øh, kan jeg hjælpe dig med noget, eller kigger du bare?
\act{FM stirrer vredt og fortvivlet på drengen, så på skiltene, så på drengen, så på publikum, så på drengen}
\says{FM}[behersker sig lidt] Nej, jeg kigger bare!
\says{FD}[forsøger at smile] Okay, øhm du må sige til hvis du får brug for hjælp. Jeg kan informere om at vi i denne uge har tilbud på peberfrugt
\says{FM} PEBERFRUGT?! Selvfølgelig er det en frugt -- det er de jo allesammen?!
\says{FD} Nej, nej, frugterne er derovre\ldots \act{pege på ``Frugt''-afdelingen} 
\says{FM} Men, det giver jo ingen mening! Der må være konsistens! Så skal i kalde den her \act{finder en agurk} for agurke-frugt! \act{finder en tomat} og ``tomat-frugt'! 
\says{FD}[gør nar af FM] lol, ja, og svampefrugter. Haha!
\act{FM skal til at overfalde FD -- opdager at han har ret. Det \emph{er} frugtlegemer!}
\says{FM}[begejstret] Ja! Lige præcis!
\says{FD}[kigger opgivende på FM] \ldots men de \emph{er} jo grøntsager!
\act{FM kan ikke kontrollere sin vrede og angriber FD med sin bog/agurk, og han bliver slået bevidstløs. Rarrrgh!}
\says{FM}[indebrændt, kigger på agurken]Det er altså et bær\ldots
\scene Derefter går han i gang med at rette på skiltene: ``peberbær'', ``jordopsvulmede'' blomsterbunde, kirsefrugt osv. Nogen må finde ud af hvad der foregår. Måske kommer der skilte ud fra bagtæppet med de rigtige ting på, indtil FM opdager ``Grønthandler''-sklitet. Der kommer et ``Frugthandler''-skilt ud, han afviser det; de har jo også svampe, nyt skilt, osv.
\scene Grønthandleren kommer ind og ser den bevidstløse fejedreng
\says{GH}[Med en tyk accent] Hva' fanden sker der her?!
\says{FM} \act{peger med agurken} Dig! Så mødes vi endelig \act{laver gåseøjne med fingrene} ``Grønthandler''!
\says{FM}[heroisk] Alt for længe har Danskerne levet i bundløs uvidenhed mens løgnen ``grøntsager'' er blevet trukket ned over øjenene på dem \ldots
\says{GH} Men, du har slået min fejedreng ned \ldots
\says{FM}[mere heroisk rant]\ldots et sølle offer! Tiden er kommet til at oplyse det danske folk, og give dem mulighed for vidensbaseret valg om de vil have frugter i deres salat!
\scene Mens FM ranter, går han også rundt og sætter nye skilte på ``frugt''-kasserne
\says{GH} \ldots  Men, jeg tvinger ingen til at lave salat -- for min skyld kan de komme rodfrugter i, hvis de vil.
\says{FM} RODFRUGTER?! Det mest blasfemiske af alle pseudonavne! Det er folk som dig, der er skyld i at folk tror på springløg(nen)! % eller noget andet sjovt, haha, springløg == forårsløg oversat fra engelsk.
\scene FM er færdig med at sætte korrekte skilte på ting, og stikker af i en røgsky! Dramatisk! Kappe! Damdamdam!


\scene Efter at FM er forsvundet står grønthandleren og stirrer lidt forvirret ud på publikum, måske måbende, hvorefter hans ansigts udtryk pludselig ændres til noget mere skummelt. Han finder telefonen frem, og ringer til undervisningsminister Sofie-Carsten, han har ikke længere nogen accent, som han havde tidligere. Måske ved man ikke hvem han ringer til endnu.
\says{GH} Fagterm-man har lige været her i butikken, og har opdaget Planen!
\says{MS} Hmm, godt nok, alt går efter planen. Igangsæt fase 2!
\scene Men hvad de ikke ved er at en mørk skikkelse har siddet ude i siden af sketchen, og afsløres som Eske Willerslev, der har hørt hele samtalen.
\says{FT}[voice-over]\ldots Men hvad de ikke vidste var, at Eske Willerslev havde gemt sig bag en af frugt-boderne, og hørt hele den hemmelige samtale!
\scene GH og FD står stille og undrer sig over hvor stemmen kommer fra?
\says{EW}\act{kommer op bag en af ``frugt''-boderne} Jeg er Eske willerslev, og jeg har gemt mig bag frugtboderne og hørt hele samtalen!
\says{FT}[voice-over] Eske stikker af med sin nye viden -- Det skal Fagtermman vide!
\says{EW} Jeg stikker af med min nye viden! 

\says{EW} Men inden Eske Willerslev stikker af\ldots \act{henvendt til publikum} har I læst min seneste artikel?
\scene Eske Willerslew deler et par artikler ud til folket
\says{FT} øøøh\ldots Eske Willerslev delte hurtigt et par artikler ud, og skyndte sig så ud for at advare fagtermman!
\says{EW}[Begejstret] Neeej, Eske havde mange artikler at dele ud!
\scene FT kommer op af lemmen, som sufflør
\says{FT}[Vredt, insisterende] Eske Willerslev indser at han har meget vigtigere ting at tage sig til, og skynder sig
\says{EW}[Let nedtrykt] Eske Willerslev har åbenbart vigtigere ting at tage sig til, og skynder sig videre
 
\scene EW forlader scenen

\end{sketch}
\end{document}