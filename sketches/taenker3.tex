\documentclass[a4paper,11pt]{article}

\usepackage{revy}
\usepackage[utf8]{inputenc}
\usepackage[T1]{fontenc}
\usepackage[danish]{babel}

\revyname{Biorevy}
\revyyear{2014}
% HUSK AT OPDATERE VERSIONSNUMMER
\version{0.1}
\eta{$2$ minutter}
\status{Ikke færdig}
\responsible{Hygge}

\title{Altså, nu tænker jeg bare højt 3}
\author{Markus}

\begin{document}
\maketitle

\begin{roles}
    \role{RE}[Jean] Revyst
    \role{ST}[AV] Stemmer
\end{roles}

\begin{props}
    \prop{Revy-t-shirt}[Person, der skaffer]
\end{props}

\scene
Scene \& stemning:
En enkelt revyst står midt på scenen, uden at bevæge sig ret meget.

Bemærkninger til Teknikken:
AV Skal have en optagelse med undertekster.

\begin{sketch}

\scene{Lys op, revysten går ud midt på scenen og stiller sig}
\says{RB} ”Altså, nu tænker jeg bare højt..”

\says{S1} ”Ja se, det går jo fint det her, du står på scenen, folk er glade”
\says{S2} (Hulkende) ”Nej, det går jo helt ad helvede til, revyen har sejlet, folk er bare for stive til at opdage det
\says{S3} Pifter. Lyder fuld ”Hvad så!?!?! Er der fest?!?!”
\says{S2} Desperat og irriteret ”Nej vi har prøvet at hutle os igennem den her sketch i halvanden time”
\says{S4} ”Hey flyt jer lige, vi arbejder faktisk her”
\scene Byggelarm, saven osv.
\says{S1} ”Ikke midt i revyen! Vi står lige og brillierer
\says{S3}  ”knuckles for helved!”
\scene Trafiklyde og en dytter
\scene Jean bevæger sig ud foran til siden på scenen og bluescreener.
\scene Næste Sketch går i gang med Jean stående stille på scenen.


\end{sketch}
\end{document}
