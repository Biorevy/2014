\documentclass[a4paper,11pt]{article}

\usepackage{revy}
\usepackage[utf8]{inputenc}
\usepackage[T1]{fontenc}
\usepackage[danish]{babel}

\revyname{Biorevy}
\revyyear{2014}
% HUSK AT OPDATERE VERSIONSNUMMER
\version{0.2}
\eta{$1$ minut}
\status{Lummer}
\responsible{Markus Back}

\title{Altså, nu tænker jeg bare højt 2}
\author{Back, Valdbjørn}

\begin{document}
\maketitle

\begin{roles}
    \role{RE}[Søren Holk] Revyst
    \role{ST}[Frederikke] Stemme
\end{roles}

\begin{props}
    \prop{Samme som Tænker 1}[Person, der skaffer]
\end{props}


\begin{sketch}

\scene Lys op, Revysten går ud midt på scenen og stiller sig 

\says{RE} Nu tænker jeg bare højt\ldots
\scene Revyst har glemt sin replik
\says{ST} Sikken skarpt lys! \ldots det er nu egentlig meget flatterende, fanger de lyse lokker ret godt.
\says{ST} Jeg er jo som skabt til scenen! Så alle kan også se min flotte, stærke krop.
\says{ST} Nej, sketchen, videre! 
\says{RE} Nu tænker jeg bare højt \ldots
\says{ST} Hold da kæft en nedringet kjole, de stirrer jo nærmest på mig!
\says{ST} Hun vil ha' mig! Og hende der, halløjsa! De vil alle samme ha' mig, der er jo ikke et sæde tørt. 
\says{ST} Der bliver godt udvalg til efterfesten! Mmmmmmmmmm \ldots
\says{ST} Åh åååh \ldots
\scene Holk går ud af scenen, imens, kommer han lidt. 
\says{RE} (Når han er bag tæppet) ÅÅÅÅÅÅÅÅRRRRHHH

\scene Lys ned 
\end{sketch}
\end{document}
