\documentclass[a4paper,11pt]{article}

\usepackage{revy}
\usepackage[utf8]{inputenc}
\usepackage[T1]{fontenc}
\usepackage[danish]{babel}

\revyname{Biorevy}
\revyyear{2014}
% HUSK AT OPDATERE VERSIONSNUMMER
\version{0.1}
\eta{$n$ minutter}
\status{Ikke færdig}
\responsible{Jean}

\title{Censor fra RUC}
\author{Valdbjørn}

\begin{document}
\maketitle

\begin{roles}
    \role{CE}[Søren Holk] Censor fra RUC
    \role{EK}[Siren] Eksaminant
    \role{VL}[Anna Aa] Vejleder
\end{roles}

\begin{props}
    \prop{Stole}[Person, der skaffer] 2 stole i nyt Nordisk Design
    \prop{Slides}[Person, der skaffer] Afslutning af speciale-forsvar
\end{props}


\begin{sketch}

En biolog er i gang med at forsvare sit speciale og har en censor fra RUC, som bliver ved med at afbryde med sine bongotrummer og abstrakt dans. Klager over at det er en individuel eksamen og at folk ikke sidder på puder i en rundkreds.
Karakteren er selvfølgelig et flyvsk begreb og skal diskuteres i plenum.

Der kan muligvis stilles endnu dummere spørgsmål i slutningen af sketchen.

\says{EK} \ldots og det er derfor at ræven klarer sig så godt i den nye danske Natur. Nogen spørgsmål?
\says{CE} God fremlæggelse, jeg vil dog lige komme med nogle ideer til dine fremtidge fremlæggelser. Ude på RUC, hvor jeg jo er professor plejer vi at side på puder i en rundkreds på gulvet. Dette åbner op for en helt unik og gribende måde at diskutere emnet på.
\says{EK} Tjo det kan jeg vel godt prøve på næste gang (forvirret).
\says{CE} skide godt!!
\says{EK} Spørgsmål til opgaven??
\says{CE} ja godt du siger det. Jeg kan se at I var to om at skrive den. Hvor er ham den anden? Jeg havde håbet på en gruppe eksamen.
\says{EK} Han kommer efter mig og snakker om den del han har lavet. Det er en ting vi gør her på KU.
\says{CE} Der kan man se, det må man jo håbe bliver ændret.
\says{EK} Okay.... Nogle spørgsmålet om Vulpes vulpes eller den røde ræv?? (lidt mere bestemt) 
\says{CE} Ja jo, jeg har taget lidt lækre noter til din meget spændende fremlægning. For det første snakkede du om at ræven er omnivor, men synes du ikke det er lidt at skære dem over en kam. Tænk på hvis en ræv måske har valgt at være vegetar. Hvordan ville dens kost se ud tror du?
\says{EK} tjo vil mene at den helst ville gå efter bær og lignende da de er let tilgængelige.
\says{CE} Men prøv at forestille dig, at du er en ræv som er vegetar. Hvad ville du spise?
\says{EK} Grøntsager tror jeg?
\says{CE} Jahh, og \ldots ?
\says{EK} Svampe?
\says {CE} Jahh, og \ldots ?
\says {EK} Bønner?
\says {CE} Jahh, og \ldots ?
\says{EK} Errm\ldots ? Tofu pølser?
\says{CE} Netop, skide godt.
\says{CE} Det næste jeg så vil høre om er om du vil have mod på at fortolke den fremlæggelse på en ny måde for mig. Dette vil være for at se om du også har forstået det du har lavet på et andet end videnskabeligt plan.
\says{EK} Hmm jo det tror jeg godt jeg kan. Hvad tænker du på?
\says{CE} Glimrende, du kender vel godt svanesøen??
\says{EK} Åååhh nej \act{hviskes}. Jo det gør jeg.
\says{CE} hvis jeg så siger ræveskoven.
\act{EK begynder at danse dumt som en ræv?}
\says{CE} BEGEJSTRET begynder at give instukser: Skide godt. Gøre det
faverigt! Gør det trekantet! Gør det pelset. Gør det cirklet!
\act{Dansen bliver meget mærkelig.}
\says{VL} Mange tak, nu tror jeg at vi har set nok.


\end{sketch}
\end{document}
