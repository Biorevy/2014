\documentclass[a4paper,11pt]{article}

\usepackage{revy}
\usepackage[utf8]{inputenc}
\usepackage[T1]{fontenc}
\usepackage[danish]{babel}

\revyname{Biorevy}
\revyyear{2014}
% HUSK AT OPDATERE VERSIONSNUMMER
\version{0.1}
\eta{$7$ minutter}
\status{Ikke færdig}
\responsible{Sofie}

\title{Knoglesammenlægningssketchen}
\author{Markus D.}

\begin{document}
\maketitle

\begin{roles}
    \role{JR}[Siren] John Renner
    \role{D}[Sofie] Direktør for SNM
    \role{A}[Hygge] SNM-arbejder 1
\end{roles}

\begin{props}
    \prop{Flyttekasser}[Dem kan ``vi'' nok selv skaffe] Til knoglerne
    \prop{Knogler fra dinosauren ``Misty''}[Person, der skaffer] Femur, humerus, kranie, komplet rygsøjle, thorax, hofte
    \prop{Koskelet i profil}[Person, der skaffer] På en papplade som kan bøjes ind halvvejs og sættes sammen med \ldots
    \prop{Chimpanseskelet i profil}[Person, der skaffer] \ldots som også kan bøjes ind halvvejs
    \prop{Girafhals med velcro}[Person, der skaffer] Til at sætte på Misty
    \prop{Strudsehals med velcro}[Person, der skaffer] Til at sætte på Misty
    \prop{Slange}[Person, der skaffer] Til at sætte på Misty
    \prop{Flamingohals}[Person, der skaffer] Til at sætte på Misty
    \prop{TRANE?!}[Person, der skaffer] Til at sætte på Misty
\end{props}

\scene
Scene \& stemning
Vi er på SNM

Bemærkninger til Teknikken:
SNM belysning

\begin{sketch}

\scene{Beskrivelse}
\scene{Lys op.}
\scene{D står meget ivrig og glad og ser på den  helt nye Diplodocus longus eller kendt som langhals.} 

\says{D} Neeeej, hvor er det fint! Den helt nye 150 mio år gamle planteædende Diplodocus longus dinosaur er kommet til SNM! 17 meter lang, og op til 16 tons! Hvor er den fantastisk!

\says{A} Se, hr Museumsdirektør! En femur! Hvor er den stor og flot!

\scene (A) sætter femur på dinosaur skelettet.

\says{A} Og se, en humerus!

\scene (A)  sætter humerus på Misty skelettet.

\scene D bliver vanvittigt oveivrig.

\says{D} NEEEEJ HVOR ER DET FLOT! [ivrig! Løber frem og tilbage] Uhhh den vil pryde SNMs haller så flot så fint. Hør, hvornår tror I vi kan få knoglerne samlet sammen?

\says{A} Med det samme! Ja! Vi skal sætte alle knoglerne på!

\scene Et stort mandigt brøl høres bag tæppet. Alle stopper med at arbejde.

\says{JR} \act{Stadig bag tæppet} STOOOP! \act{kommer ud på scenen.}

\says{JR} Hvad foregår der her? Hvad er alt dette for noget postyr?

\says{D} John Renner! Dekan for det Naturbiovidenskabelige fakultet! 

\says{JR} Det Natur - OG – Biovidenskabelige fakultet!

\says{D} Undskyld... og goddag! Ønsker Dekanen at se vores nyeste tilføjelse til samlingen her på SNM??

\says{JR} SNM, det kender jeg ikke noget til! Jeg er John Renner – Dekan for det Natur- OG – Biovidenskabelige fakultet!

\says{D} Jamen John Renner! Du skal da se vores nye dinosaur! Se alle de fine knogler der er kommet, vi har fået over 200! 

\says{JR} 200!!! Hør lige engang her – så meget plads har Det Natur – OG – Biovidenskabelige fakultet slet slet ikke. Mit motto har jo altid været SYNERGI og ERHVERVSRELEVANS gennem SAMMENLÆGNING.

\says{D} Jamen...

\says{JR} Så De KAN vel nok se at vi ikke kan have 200 knogler fra én ENKELT studieretning af dinosaurer! Det er HELT uden synergi og erhvervsrelevans!

\says{D} Jamen det er jo fra et enkelt individ.

\says{JR} Jeg har hørt nok! Næh nej. Se nu blot her. \act{Kigger tilfældigt omkring på scenen hvor der står skeletter.}

\says{JR} Vi tager disse knogler - \act{Griber ko-skelet (se illustration).}

\says{D} Jamen de er fra en ko??

\says{JR} Glimrende, ganske erhvervsrelevant, tænk dog på husdyrvidenskaberne. Disse SAMMENLÆGGER vi med disse her \act{griber chimpanse-skelet}

\says{D} Fra en abe!?

\scene JR sætter de to skeletter sammen ved at bøje pappladerne ind på hvert skelet og sætte dem sammen, så de bliver til en halv ko og en halv chimpanse. Se illustration.


\says{JR} Fantastisk, en ko med hænder! Så kan koen jo malke sig selv. Fantastisk synergi! Sådan, SAMMENLÆGNING! Sådan gør man det!

\says{D} Jamen hvad så med Diplodocus longus, vores nyanskaffede langhals?

\says{JR} Bevares! Husk nu hvad jeg har fortalt – SYNERGI og ERHVERVSRELEVANS gennem SAMMENLÆGNING. Se nu her! Så stor en krop har plads til mange flere halse.

Griber forskellige langhalsede dyr (giraf, struds, slange, flamingo, etc.) og sætter dem på dinosaurkroppen.

\says{JR} Sådan! All-in-one! Win-win! Det handler om prioritering!

\says{D} Men John... De passer jo ikke sammen. Det ligner en – HYDRA!

\says{JR} Hydra? Fantastisk! Sikke et vidunder, masser af pladsbesparelse! Og ganske nyt, vi kan udgive en herlig masse publikationer!

\says{D} Jamen man kan da ikke bare skabe et græsk mytologisk fabeldyr.

\says{JR} NEEEJ! \act{Sparker alle halse af.}

\says{JR} Vi skal satme ikke ha' græske dyr her, de har kun bankerotter!

\says{D} For fanden John, nu har vi kun en kæmpe bunke knogler. Det kan man jo ikke lære noget af, tænk på alle de nye biologistuderende som skal studere dyrene.

\says{JR} Så de kan de jo bare selv samle dem. Ja det er en god ide, learning by doing, så kan vi jo slå det sammen med eksamen i OD. Og så har jeg jo slet ikke brug for jer, sådan en bunke knogler kræver næsten ingen vedligeholdelse. I er hermed afskedigede. Se det er sammenlægning!


\end{sketch}
\end{document}
