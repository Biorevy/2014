\documentclass[a4paper,11pt]{article}

\usepackage{revy}
\usepackage[utf8]{inputenc}
\usepackage[T1]{fontenc}
\usepackage[danish]{babel}

\revyname{Biorevy}
\revyyear{2014}
% HUSK AT OPDATERE VERSIONSNUMMER
\version{1}
\eta{$n$ minutter}
\status{Næsten færdig}
\responsible{Thomas BT}

\title{Byggerisketchen}
\author{Jennie, Danny, Thomas BT}

\begin{document}
\maketitle

\begin{roles}
    \role{FB}[Nathalia] Frustreret Biolog
    \role{PB1}[Sus] Positiv biolog
    \role{PB2}[Charlotte] Positiv biolog
    \role{SS}[Frederikke] Sød Studerende
    \role{GF}[Norlin] Grædende forsker
    \role{EA}[Danny] Entusiastisk Arkitekt
    \role{AV1}[Valdbjørn] Arkitektens Ven 1
\role{AV2}[Gunn] Arkitektens ven 2
\role{AV3}[Markus Back] Arkitektens ven 3
    \role{HV1}[Siren] Håndværker 1
    \role{HV2}[Mia] Håndværker 2
\end{roles}

\begin{props}
    \prop{Bord}[Person, der skaffer] Måske et bord.
    \prop{Stol}[Person, der skaffer] Måske en stol.
    \prop{Mikroskop}[Person, der skaffer] Måske et mikroskop.
    \prop{Værktøj}[Person, der skaffer] En boremaskine
    \prop{Skygge-maskineri}[Thomas BT] Pap-ting, der kan kaste skygger op på skygge-skærmen
    \prop{Vask}[Person, der skaffer] Vask. Skal sættes fast i gulvet
    \prop{Planen}[Hvem skaffer] Et stykke papir, evt. er det bare et skattekort, med et kryds. 
     \prop{Byggeri-lyde}[Hvem skaffer] Noget der kan larme, og afspilles af AV
\end{props}


\begin{sketch}

\scene Celledelingsdansen slutter idet Byggerisketchen starter

\scene Håndværkerne skal interagere med folk på upassende tidspunkter. Der skal være punchlines og rants mens der larmes, haha, meget sjovt!
Der skal være forskellige frustrerede folk (forskere, studerende, rengøringspersonal,
Sketchen starter når skyggedansen slutter. Der sidder en frustreret biolog og er træt af at alting ryster. Håndværkerne lider måske lidt af mindreværdskomplekser, måske rimelig passiv aggressive desuden går lyset ned flere gange I løbet af sketchen.  

\scene Vi skal have nogle sceneninjaer til at tage skygge-rammen ned. 

\says{FB} ARGHHHH! \act{rystende vredes udbrud} Jeg kan overhovedet ikke se hvad der sker! Det hele vælter rundt, hvordan skal jeg nogensinde få lavet noget!

\scene FB rejser sig vredt og macherer ud på gangen (midten af scenen, hvor han før måske sad ude i siden med sit mikroskop, eller noget)

\scene Ude på gangen kommer en sød studerende gående, lidt flirtende smiler han/hun til FB, men da bygningslarmen starter går hun/han amok og råber og skriger (mutet self.) larmen stopper inden hun/han er færdig

\says{SS} \ldots DET ER SÅDAN MAN SLÅR BØRN IHJEL!
SS falder ned igen, og smilende går videre, FB ved ikke helt hvad FB skal gøre af sig selv, men prøver med et anspændt smil.
\says{FB} Nåå \ldots såå du er heller ikke vildt glad for byggeriet?
\says{SS} Nej, her den anden dag sad jeg til en forelæsning, hele bygningen rystede så meget at jeg faldt af sædet. Mange studerende går grædende hjem fra undervisning med galloperende migræne \ldots

\scene 2 håndværkere kommer ind stille og roligt, med håndværker tools og en bajer I hånden.

\says{HV1} Undskyld mig, vi skal lige bore et hul I gulvet her, så vi larmer nok lige lidt.
\says{FB} Jamen, det er jo midt på gangen? Hvorfor skal I bore her?
\says{HV2}[smilende] Nå, men det står lige her I planen! \act{HV2 viser Planen frem -- derefter går de i gang med at bore I gulvet, og SS går sin vej.}
\says{HV1} Hold lige den her \ldots \act{Rækker sin bajer ud til FB.}

\scene En forsker kommer ind og lyset slukkes [lys ned]. Forskeren begynder at græde.

\says{GF} Åh nej klynker han/hun Nu skal jeg starte helt for fra, en hel måneds arbejde spildt, igen.

\says{HV2} \act{ringer med sin mobil} Hey Torben, tænd lige for strømmmen igen, vi er lige i gang med at larme
\scene [lys op]
\says{GF} Åhhhuhuh \ldots Er I klar over hvor mange forskningsmidler I lige har spildt?!
\says{HV1} Øhh det har vi slet ikke noget at gøre med!
\says{HV2} \act{står og kigger på planen} Næ, forskningsmidler står der ikke noget om i planen.
\scene GF hulker videre, og måske bare går grædende ud.

\scene Håndværkerne er færdige med at bore, og går også ud. Lidt efter kommer de tilbage med en vask
\says{FB} Hvorfor skal der stå en vask midt på gangen?
\says{HV2} Det står jo i planen!
\says{FB} Jamen, skal I ikke sætte den fast??
\says{HV1}[fornærmet] Ligner jeg måske en VVS'er? Nu må du simpelthen holde op. 
\says{FB} Jamen, det giver jo ingen mening at den står her.
\says{HV1}  Næ, det er jo egentlig rigtig nok, \act{Infernalsk larm} men det står jo altså i planen!

\scene EA og venner kommer brasende ind!
\says{EA}[entusiastisk!] Ej, hvor er det fedt, der er kommet hul i gulvet! Det har jeg ventet på så længe! I kan jo tydelig se hvordan jeg har dekonstrueret vores forståelse for den moderne arbejdsplads \ldots
\says{FB} Men det er jo en vask i gulvet! Ude på gangen?
\says{EA} Ja, præcis, vi skal rive væggene ned [Håndværkerne mærker på væggen], altså metaforisk! Du har nemlig ingen fornemmelse for om du er ude eller inden for dette rum!
\scene Imens står vennerne og nikker/ klapper/ tager billeder/selfies/ sniffer coke.
\act{EA trækker sit lille håndspejl frem, og hælder en ordentlig pose coke derudover, hvor efter han stikker hele krydderen ned i det!}
\says{EA}  Nå, drenge -- Skal vi komme videre og se på mit fremtidige mesterværk?! \act{Vennerne knipser fornemt og tager selfies, og går videre.}
\says{FB}[taler til håndværkerne] Ja, der blev alle fordommene da vidst lige bekræftet, det minder mig om en vitighed min onkel altid fortalte: En arkitekt, en præst og en nymphoman går ind på en bar.. \act{LAAAAAAARM}.. og så kan man se hvor svært det er med store fødder..
\scene Den positive biolog kommer ind på scenen, PB er venner med FB. FB sukker højlydt.
\says{PB} Hvad sukker du da sådan over?
\says{FB} Jeg bliver bare så stresset af al den byggeri!
\says{PB} Nå, det skal du da ikke tage så tungt, nogen gange skal man se på glasset som halv fuldt eller halvt tomt.
\says{FB} Jeg er ret sikker på, at det glas er gået i stykker!
\says{PB} Hvad mener du?
\says{FB}  Ja, alting ryster jo og falder på gulvet!
\says{PB} Nå nå, måske en lille sang kunne muntre dig op?

Byggerisangen begynder!

\end{sketch}
\end{document}
